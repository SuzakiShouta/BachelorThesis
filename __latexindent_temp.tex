\thispagestyle{myheadings}
\chapter{関連研究}
\label{sec:format}

% 本章では本研究の関連研究について述べる.
% 2.1では,スマートフォンを用いたセンシングとジオフェンシングの関連研究について述べる.
% 2.2では,1章で説明したクラウドセンシングの関連研究について述べる.
% 2.3では,クラウドセンシングの基盤となるプラットフォームの運用例や関連研究,それらの協力者に対するモチベーションの維持・向上方法について述べる.

\section{クラウドセンシングに関する研究}
\label{sec:format_thesis}
クラウドセンシングを利用して多くの人々からデータ収集し,推定や分析をする研究はいくつかある.
西村らの研究\cite{ura}は,スマートフォンの加速度センサとマイクからスマートフォン保持者の歩行動作と周辺の雑踏音をそれぞれセンシングし,端末周辺の混雑状況の推定を行っている.
混雑状況の推定の研究としては,バス内の混雑状況を加速度データと角速度データから推定する研究\cite{hoso}もある.
これは混雑時にバス利用者が他のバス利用者を避けるために,体を横に捻ったり,肩をすぼめて移動したりする回避動作のデータを収集し,そのデータから混雑状況を推定するというものである.
このような場合,収集データ量が多ければ多いほどより詳しく混雑状況を推定できる.
朴らの研究\cite{paku}では一般の自動車利用者から加速度センサなどのモーションセンサを用いてデータを収集し,凍結や舗装路などの路面状態や平坦やくぼみなどの路面形状の推定を行っている.

これらの研究ではクラウドセンシングシステムの開発などには大きなコストがかかると考えられる.
依頼者がクラウドセンシングを実施するためには,協力者専用のセンシングスマートフォンアプリケーションや収集したデータを管理するサーバなどを開発する必要がある.
協力者が依頼者の友人または研究仲間の場合,操作方法は口頭で補えばよいため,デザインや仕様にこだわる必要はない.
しかし,協力者が赤の他人である場合,アプリケーションの操作方法を初めてでも理解しやすく簡易的にするなど工夫する必要がある.
また,センサなどの種類やサンプリングレートをあらかじめ指定してアプリケーションを作成してしまうと,途中で変更したい場合に再インストールなどしなければならず,依頼者・協力者共に負担がかかってしまう.
アプリケーション内でセンサなどの種類やサンプリングレートを変更する機能があっても,協力者の操作が必要であり,負担は増えてしまう.
そのため,クラウドセンシングを利用してデータ収集したい場合は,プラットフォームがあると非常に便利である.
依頼者としては,研究毎にセンシング専用のスマートフォンアプリケーションを作成・配布する必要がなくなるため,それらに費やしていた時間や手間の省略が可能となる.
また,一定の仕様やデザインはプラットフォームとしてあらかじめ決まっているので,一から考える必要はない.
協力者としては,研究別のスマートフォンアプリケーションをインストールする手間が省け,研究毎に使い分ける必要もなくなる.
そして,協力者のスマートフォンアプリケーションの利用により,協力者を他のセンシングに誘導も可能であるため,多くの協力者獲得にも繋がる.


\section{クラウドセンシングプラットフォームに関する研究}
\label{sec:style}
実際に運用を行っているクラウドセンシングプラットフォームとして,Ohmage\cite{Tangmunarunkit}\cite{ohmage}やAWARE\cite{Ferreira}\cite{AWARE}などがある.
Ohmageはセンサデータ収集だけではなく,質問を作成し,協力者にテキストや選択肢,動画,画像,音声といった方法で回答し,収集が可能である.
また,収集したデータは可視化・分析でき,独自の統計分析のためにデータのエクスポートも可能である.
AWAREとは,依頼者向けにセンサー計測によるモバイルコンテキスト情報の計測,推測,ログ記録,共有に特化したプラットフォームである.
AWAREはプラグインが可能であるため,AWAREで作成したアプリの拡張・機能の追加をし,依頼者独自のクラウドセンシングが実施できる.

クラウドセンシングは協力者の確保が非常に重要であるため,様々な方法でモチベーションを向上・維持させる必要がある.
協力者のモチベーションを向上させる方法として金銭的インセンティブ,非金銭的インセンティブがある.
金銭的インセンティブとは,お金などの報酬である.
金銭的インセンティブは,協力者を獲得し,協力者のモチベーションを維持するには非常に有効である.
金銭的インセンティブを用いたクラウドセンシングプラットフォームとしては,LiveLabs\cite{jaya}が挙げられる.
多くの協力者を獲得できるものの,報酬には限界があり,協力者が多くなればなるほど依頼者側の負担が大きくなる.
また,多くなった協力者を保持するための費用も多くかかる.
負担が多いからといって報酬を無くしてしまうと,協力者のモチベーションは下がってしまい,センシングに協力的ではなくなる可能性がある.
一方で,非金銭的インセンティブとは,お金などの報酬の代わりとして楽しさや体験を報酬としている.
非金銭的インセンティブを提供する手法として,ゲーミフィケーションが挙げられる\cite{ara}.
ゲーミフィケーションとは,ゲームの構造や考え方をゲームとは異なった分野に組み込んでゲーム化するといった意味である.
ゲーミフィケーションを用いた例として,河中らの研究\cite{kawa}が挙げられる.
河中らのプラットフォームは観光情報収集を目的としている.
ゲーミフィケーションの内容は,特定のエリア内を歩き回る「エリアミッション」と特定の場所で写真撮影を行う「チェックインミッション」があり,ミッションをクリアすると付与されるポイントで協力者同士競い合うといったものである.
ポイントに集めている間に加速度や角速度などのセンサの値を収集し,5秒ごとにサーバに送信している.
「もっとポイントを獲得したい」という思いがセンシングのモチベーションの向上に繋がるため,協力者維持には有用である.
モチベーションを生成するために,金銭的インセンティブと非金銭的インセンティブを柔軟に選択できる松田らのクラウドセンシングプラットフォーム\cite{matsu}もある.
これは,デフォルトとしてモチベーション生成方法が決まっている訳ではなく,依頼者が決める.
ポイントやレベル,バッジの有無や付与条件,ランキングの付け方などのゲーミフィケーションや,金銭的インセンティブなどをセンシングに応じて設定できる.
このプラットフォームはParmoSense\cite{Parmo}として既に運用されている.
しかし,モチベーションを保つために,必ずしもインセンティブが必要であるとは限らない.
先述のOhmageやAWAREは,クラウドセンシングの依頼者のシステム開発コストの削減を主に目的としたシンプルなプラットフォームである.
そのため,協力者を多く集め維持するといった部分は標準機能としては備わってはいない.
その場合,協力者のモチベーションは「センシングに協力したい,研究に貢献したい」という気持ちだけである.
例えば,坂村らのまちづくりのためのプラットフォーム\cite{mina}は,「町や地域の問題を解決したい」や「まちづくりを促進したい」といった地域の人々の思いやボランティア精神でモチベーションを保てている.
クラウドセンシングの目的やどのように社会や暮らしに還元されるかが明確であれば,インセンティブなしでも協力獲得の可能性はある.


\section{クラウドセンシングで使われるセンシング端末に関する研究}
\label{sec:format_abst}
クラウドセンシングのセンシング端末として様々な端末が使用されている.
例えば,スマートフォンが使用されている\cite{ura}\cite{hoso}\cite{Tangmunarunkit}\cite{ohmage}\cite{Ferreira}\cite{AWARE}.
クラウドセンシングのセンシング端末にスマートフォンを利用するメリットとして,スマートフォンは普及率が高く,新たにセンシング端末を確保,配布,改修する必要がない点が挙げられる.
また,クラウドセンシング専用のモバイルアプリを作成すれば,インターネットを介して世界中の人がクラウドセンシングに参加できる.
しかし,デメリットとして,協力者がクラウドセンシングに協力しようとすると,協力者がそのクラウドセンシングに対応したアプリケーションをそれぞれインストールする必要がある.
また,スマートフォンは常に携帯している場合が多く,センサの種類や時空間によってはプライバシを侵害してしまう.

スマートフォンを使用しない例として,市販の環境センサや,専用に開発されたものがある\cite{paku}\cite{weko}\cite{amedas}\cite{S-net}.
クラウドセンシングのセンシング端末にスマートフォンを使用せず,市販の環境センサや専用に開発したものを使用するメリットとして,長時間のセンシングや大規模なセンシングなど目的に沿ったセンサが使用できる点が挙げられる.
デメリットとして,イニシャルコストとランニングコストがかかる点が挙げられる.



% Local Variables: 
% mode: japanese-LaTeX
% TeX-master: "root"
% End: 
