\chapter{特定の時空間への進入時に自動センシングするアプリケーション}
\thispagestyle{myheadings}
本章では特定の時空間への進入時に自動センシングするアプリケーションについて述べる.
\ref{lavlusReq}章ではまずラヴラスのモバイルアプリケーションの要求仕様を定義する.
\ref{myApp}章では特定の時空間への進入時に自動センシングするアプリケーションの実装に述べる.
\ref{myApp_STF}章では時空間フェンシングの実装について述べる.
\ref{myApp_notify}章ではセンシング依頼通知の実装について述べる.
\ref{myApp_sensing}章では自動センシングの実装について述べる.


\section{ラヴラスのモバイルアプリケーションの要求仕様}
\label{lavlusReq}
ラヴラスのモバイルアプリはセンシングプロジェクトダウンロード,時空間フェンシング,センシング依頼の承諾,自動的にセンシング,Wi-Fi環境下で自動的にアップロードができる必要がある.
協力者の物理的コストを軽減させるために,協力者への通知と協力者自身の操作の低減や,端末のデータ通信量を圧迫しない必要がある.
センシングプロジェクトをダウンロードする時,端末の通信量とデータ容量を圧迫しない為に,協力者が参加する可能性があるセンシングプロジェクトのみダウンロードする必要がある.
また,センシングが終わった後,Wi-Fiに接続している時に自動でアップロードされる必要がある.
協力者の物理的コストを軽減させるために,協力者へ通知と協力者自身の操作を最小限に抑える必要がある.
依頼者の制作したセンシングプロジェクトに参加する可能性が高い協力者のみにセンシング依頼通知を送る必要がある.
例えば,時空間に近いもしくはすでに進入している場合である.
また,一度センシング依頼に承諾,拒否した場合そのセンシングプロジェクトから通知は発行されない.
時空間に進入している間,バックグラウンドで自動でセンシングされる必要がある.
協力者の心理的コストを軽減するために,協力者はすでに承諾したセンシングプロジェクトへの拒否や,既にアップロードしたセンサデータに削除申請が出せる必要がある.
本アプリはクラウドセンシングプラットフォームとして,多くのセンサに対応する必要がある.
また,アップロードされるセンサデータは協力者のプライバシを侵害しない為に,匿名化及び抽象化する必要がある.

% 協力者が本アプリを起動、もしくは起動してから一定時間毎にサーバからセンシングプロジェクトをダウンロードする。
% あいまいな位置情報
% 協力者が時空間に進入した場合、通知が発行され、センシング依頼画面が立ち上がる。センシング依頼に承諾した場合、時空間に進入している間、バックグラウンドで自動でセンシングされる。
% センシングが終わった後、Wi-Fiに接続している時に自動でアップロードされる。
% 協力者はすでにセンシングに承諾したセンシングプロジェクトにもセンシング拒否ができる。
% また、送信したセンサデータに削除申請ができる。



\section{特定の時空間への進入時に自動センシングするアプリケーションの実装}
\label{myApp}
依頼者の制作したセンシングプロジェクトに対応したセンシングをするためにモバイルアプリを実装する.
作成するモバイルアプリの候補としてiOSとAndroidアプリがある.
本研究で作成するアプリはクラウドセンシングプラットフォームとして多くのセンサに対応する必要がある.
そのため,iOSと比べて多くのセンサを取り扱えるAndroidアプリを作成した.

本アプリは4.1章で述べた内、時空間フェンシング、センシング依頼の承諾、自動的にセンシングのみ実装した。

\begin{figure}[tbh]
    \centering
    \includegraphics[width=16cm]{img_myApp.png}
    \caption{特定の時空間への進入時に自動センシングするアプリケーションの全体図}
    \label{fig:myApp}
\end{figure}

実装したアプリの全体図を図\ref{fig:myApp}に示す.


\subsection{時空間フェンシングの実装}
\label{myApp_STF}
ジオフェンシングには緯度経度,BLEビーコン,Wi-Fiなどが使われる.
ラヴラスではジオフェンシングを確実に認識する必要がある.
そのため,今回は依頼者と協力者が視覚的に認識しやすい緯度経度を採用した.
% 時空間に進入しているかの判定のため,一定間毎に位置情報を現在時刻を取得する.

ジオフェンスが緯線,経線に並行な線のみでできている長方形の場合,条件分岐を用いてジオフェンシングが可能である.
現在位置の緯度がジオフェンスの最北端よりも低く最南端よりも高い,また経度が指定エリアの最東端よりも低く最西端よりも高い場合,ジオフェンス内にいると判定する.
この条件分岐を用いると,正常にジオフェンシングできるのは緯度経度と平行な辺のみで構成されている矩形のみとなる.
ラヴラスのユースケースとして,特定の施設や建物などを対象とする場合があり,多くの施設や建物は緯度経度に並行ではない.
緯度経度に並行ではない場合,条件分岐では正確なジオフェンシングができない.
例えば,ジオフェンスがひし形の場合を想定する.

\begin{figure}[tbh]
    \centering
    \includegraphics[width=16cm]{img_polygon_1.png}
    \caption{条件分岐を用いたジオフェンシング}
    \label{fig:polygon_1}
\end{figure}

\begin{figure}[tbh]
    \centering
    \includegraphics[width=16cm]{img_polygon_2.png}
    \caption{点の内外判定を用いたジオフェンシング}
    \label{fig:polygon_2}
\end{figure}

正常なエリア判定としては,図\ref{fig:polygon_1}左の通り現在地(赤丸)にいた場合はエリア内に進入していないため,通知は送られないはずである.
しかし,条件分岐を用いたエリア判定の場合は,ひし形の外側にひし形の頂点を辺の中心とした矩形の判定エリアが形成されるため,図\ref{fig:polygon_1}右のように現在地がエリア内にいると判定され,誤って通知が送られてしまう.
このように,条件分岐を用いると自ずとジオフェンスが矩形になってしまう.

ジオフェンスが複雑な矩形な場合に対応するためにポリゴンの内外判定アルゴリズム\cite{naigai}を使用する.
点の多角形に対する内外判定を図\ref{fig:polygon_2}に示す.
点の内外判定とは,まず点(青丸)から多角形に対して線を引く.
この時引く線は直線であれば,どの方向に引いても良い.
その線と多角形との交点(赤丸)の数が奇数個であれば多角形よりも内側にいる,偶数個であれば多角形よりも外側にいると判定できる.
この場合,丁度線と多角形の辺が重なったり,線と多角形の頂点と重なると,誤った判定を行ってしまう.
しかし,緯度経度は小数点以下が7桁もあり,位置情報は変化し続けるため,丁度重なる場合は今回考えないものとしている.
依頼者が図\ref{fig:polygon_2}のようにどれだけ複雑な多角形のエリアを設定しても,内部か否かは判定可能である.
今回は実装は行っていないが円のようなエリアでも判定は可能である.
また,今回はGPSのみを用いたエリア判定のため,平面的なエリア判定しか行えない.
1階や2階などの立体的なエリア判定は今後の課題とする.

確実に時空間に進入した場合のみセンシングする,時空間に進入する可能性が高い協力者にセンシング依頼通知を発行するなど,様々なシチュエーションに対応するため,時空間の拡大と縮小が可能なマージンを実装した.

\begin{figure}[tbh]
    \centering
    \includegraphics[width=16cm]{img_margin_1.png}
    \caption{マージンの図}
    \label{fig:polygon_2}
\end{figure}

ジオフェンスが緯線,経線に並行な線のみでできている長方形の場合,加算と減算を用いてマージンの実装が可能である.
ジオフェンスを拡大する場合は,最北端の緯度と最東端の経度に加算し最南端の緯度と最西端の経度に減算する.
ジオフェンスを縮小する場合は,最北端の緯度と最東端の経度に減算し最南端の緯度と最西端の経度に加算する.
しかし,本アプリはジオフェンスが複雑な矩形に対応する必要がある.


% 緯度経度の距離の話,複雑な矩形の話,


% 複雑な矩形のマージンに対応できるように端末にマージンを設けた。

\subsection{センシング依頼通知の実装}
\label{myApp_notify}
協力者が時空間に進入するとセンシング依頼の通知が発行される。
協力者が安心してセンシングに協力できるように

\subsection{自動センシングの実装}
\label{myApp_sensing}
協力者が時空間に進入し、センシング依頼に承諾している場合、バックグラウンドで自動でセンシングされる。
クラウドセンシングプラットフォームとして多くのセンサと自由な周波数に対応した。
% ジオフェンシングの境界付近かつ,位置情報が不安定になると進入,退出の判定を繰り返してしまう.


% Local Variables: 
% mode: japanese-LaTeX
% TeX-master: "root"
% End: 
