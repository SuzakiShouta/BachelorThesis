\chapter{はじめに}
\thispagestyle{myheadings}


\section{研究背景}
\label{sec:schedule}

近年,高機能センサを備えたスマートフォンが増加している.
内蔵されているセンサの例として,加速度や角速度,磁気,気圧,Wi-Fi,BLE,音などが挙げられる.
これらのセンサは,正確で高精度な測定データの取得が可能なため,スマートフォンの3次元での位置や動作,周囲の環境のモニタリングが可能である.
例えば,ユーザの現在位置や動き,ゲームでは,スマートフォンを振る・傾けるなどの複雑な操作や動きの測定が内蔵センサによって可能となる.
追加モジュールによっては,花粉やPM2.5,風速や湿度などの計測も可能となる.

このスマートフォンのセンシング能力を活かす試みとして,クラウドセンシングがある.
% クラウドセンシングの説明図を図\ref{cloudsensing}に示す.
クラウドセンシングとは「群衆の持つスマートフォンなどの携帯端末に内蔵されたセンサを用いて低コストで大規模にセンサデータ取得し,そこから実世界の様相を把握するための方法論\cite{weko}」である.
クラウドセンシングはクラウドソーシングとセンシングを掛け合わせたものであり,参加型センシングとも言われる.
スマートフォンの登場以前は,センサデータを多くの人から取得するのは現実的ではなかった.
クラウドセンシング専用のデータ収集端末を開発しても,収集したデータの共有は困難であった.
スマートフォンは文字入力,センシング,通信機能など様々な機能を備えているため,クラウドセンシングを行うための端末として非常に適している.
スマートフォンの登場によって,クラウドセンシングという考え方が進んだと言っても過言ではない.
クラウドセンシングの技術的背景としては,スマートフォンの内蔵センサの小型化・高性能化が挙げられる.
センサの精度は初期のものと比べ,格段に良くなっている.
また,どんどん小型化・スマート化し,電力消費量も少なくなっている.
もう1つの技術的背景として,通信速度の高速化が挙げられる.
現在は4G(第4世代移動通信システム)が主流となっているが,2020年には5G(第5世代移動通信システム)のサービスが開始された.
5Gの通信速度は4Gの20倍,遅延は4Gの10分の1,4Gの10倍のデバイスに接続が可能となる\cite{5G4G}.
クラウドセンシングの社会的背景としては,スマートフォンの急速な普及が挙げられる.
2007年にApple社から「iPhone」,2008年にはAndroid端末が発売され,全世界に「スマートフォン」が爆発的に普及されていった.
日本でも主流は「フィーチャーフォン」から「スマートフォン」に変化した.
2010年はスマートフォンの世帯保有率は9.7%\cite{soumu}で,まだ全体的な普及はしていなかった.
この時点ではクラウドセンシングに協力するしないにかかわらず,まずクラウドセンシングに協力するためのスマートフォンを所持しているかしていないかで対象者が振り分けられる.
つまり,対象者の母数が圧倒的に少なかった.
クラウドセンシングは収集できるデータ数が多ければ多いほど価値が高くなるため,スマートフォンを所持していない2010年頃はクラウドセンシングを利用するには適していなかった.
2019年にはスマートフォンの保有率は世帯では83.4%と2010年と比べ約9倍,個人では67.6%\cite{soumu}となり,半数以上の人がスマートフォンを所持するようになっている.
所持している人全員がクラウドセンシングに協力してくれるとは限らないが,明らかに母数は増えているため,現在はクラウドセンシングを利用するのに適している.

クラウドセンシングは幅広いデータ収集かつセンシングコストを削減できるため,様々な研究で採用されている.
例えば,街頭の明るさ調査,観光スポット調査,騒音調査,温度調査などの研究が進められている\cite{liu}.
クラウドセンシングは研究だけではなく,自治団体や地域施設の管理者など,様々な人にとって有効になりうる可能性がある.
例えば先述の騒音調査や公害物質調査などは,自治団体にとって地域の住みやすさ改善のための重要なデータとなり,そのデータを基によりよい地域や町づくりが可能となる.
また,公園管理者にとってはいつどのような場所に人が集まるのか,危険な箇所で子供が遊んでいないか,といった施設の安全確保に有効なデータを収集できる.

しかし,クラウドセンシングプラットフォームにはいくつかの課題があり,それを解決するために我々は時空間フェンシングに基づくクラウドセンシングプラットフォーム「ラヴラス」を提案した.
クラウドセンシングのプラットフォームを作成し,クラウドセンシングの容易利用と多様なデータ収集ができるようにして,研究や調査におけるコスト(時間・費用・手間)を大幅に軽減する.
また,時空間フェンシングを提案し,クラウドセンシングを利用してセンサデータを集める人(以下,依頼者)はセンシングする範囲の容易な定義,クラウドセンシングに協力してセンサデータを提供する人(以下,協力者)はセンシングされている時空間の明確な認識を期待する.
ラヴラスを利用する場合,依頼者は専用のWebアプリケーション,協力者は専用のモバイルアプリケーションを使用する.
本研究はラヴラスのモバイルアプリケーションに関する研究である.


% 近年,高機能センサを備えたスマートフォンが増加している.
% クラウドセンシングは幅広いデータ収集かつセンシングコストを削減できるため,様々な研究で採用されている.
% しかし,クラウドセンシングにはいくつかの課題があり,それを解決するために,我々は時空間フェンシングに基づくクラウドセンシングプラットフォーム「ラヴラス」を提案した.
% % プラットフォームにして,クラウドセンシングの容易利用と多様なデータ収集ができるようにして,研究や調査におけるコスト(時間・費用・手間)を大幅に軽減する.
% % 時空間フェンシングを提案して依頼者はセンシングする範囲を定義しやすく,協力者はセンシングされている時を明確に.
% 本研究はラヴラスのモバイルアプリケーションに関する研究である.
% クラウドセンシングプラットフォームを作ろうとしていて,自分はモバイルアプリ開発をしている.
% データ収集をしたい人(依頼者)はwebアプリから,センシングに協力する人(協力者)はスマホアプリから
% 依頼者が本プラットフォームを利用してセンシング依頼(センシングプロジェクト)を作成する.

\section{クラウドセンシングの課題}

クラウドセンシングの課題として,専用システムの開発コスト,センシングによるプライバシの侵害,適切なセンサデータの確保,協力者のモチベーション維持などが挙げられる.
依頼者がクラウドセンシングを利用するためには,協力者のセンサデータを収集するための専用アプリや収集したセンサデータを管理するための専用サーバが必要となる.
自治体や地域施設の管理者などシステムの開発知識がない人にとっては自ら開発するのは困難であり,業者に委託するにも費用や手間などが大きくかかる.
また,知識のある研究者にとっても,システムの開発という研究の本質からずれた作業をしなければならないため,研究の速度は下がってしまう.
そのため,依頼者のクラウドセンシングに対するハードルは高くなっている.

依頼者の知識不足により,本来扱ってはいけない協力者のプライバシを侵害するセンサデータを集めてしまったり,協力者にセンシングがプライバシを侵害する危険性を説明しきれない可能性がある.
例えば,依頼者が個人情報にあたるセンサデータを取り扱う可能性を考慮できず,本人の同意のないまま音センサやイメージセンサで個人の氏名や生年月日等が記録されたとする.
協力者がそのセンサデータをアップロードした場合,依頼者は個人情報保護法に違反する可能性がある.

クラウドセンシングで協力者から集めたデータのクオリティが依頼者の要求するレベルに達しない場合がある.
クラウドセンシングではセンシングに慣れている人がセンシングされるわけではないため,協力者が依頼者の期待しているセンサデータを提供できるとは限らない.
また,協力者が足りず,センサデータが依頼者の必要とする量に満たない場合もある.

クラウドセンシングはセンサデータを提供する協力者が必須であるが,協力者に対するディスインセンティブ要素が多い.
協力者に対するディスインセンティブ要素として,第三者に対するデータ提供の不安,プライバシ侵害の危険性などがある.
協力者にとってセンサデータの提供そのものにインセンティブ要素はなく,依頼者がインセンティブを用意しなければ協力者の獲得は難しい.

ラヴラスが対象とする課題は,専用システムの開発コスト,センシングによるプライバシの侵害,協力者のモチベーション維持である.
クラウドセンシングにラヴラスを利用した場合,依頼者は専用システムの開発及び運用をする必要がなくなる.そのためコストの大幅な軽減ができる.
協力者は依頼者の情報やセンシングされる時空間を適切に認識し,センシングに協力すると判断した場合のみセンシングされる.
協力者は依頼者の提示した情報に少しでも不信感を覚えたらセンシングに拒否できる,
また,協力者がアップロードするセンサデータは協力者のプライバシを侵害しないように匿名化及び抽象化される.
ラヴラスではインセンティブ要素を増加させず,ディスインセンティブ要素を軽減する.そのため,インセンティブ要素と組み合わせられる.


% 依頼者側の課題として,サーバやアプリなどの専用システムの開発にかかるイニシャルコストやランニングコストが挙げられる.
% また,依頼者の知識不足により,本来扱ってはいけない協力者のプライバシを侵害するセンサデータを集めてしまったり,協力者にセンシングがプライバシを侵害する危険性を説明しきれない可能性がある.
% クラウドセンシングで協力者から集めたデータのクオリティが依頼者の要求するレベルに達しない場合がある.
% 協力者側の課題として,センサデータの提供にはディスインセンティブ要素が多い点が挙げられる.

% クラウドセンシングにはセンサデータを提供してくれる協力者が必須.
% 例えば協力者はいつセンシングされているかわからない.どのようにセンシングされているかわからない.
% プライバシを侵害されるような危険なセンシングをされている可能性がある.
% 本システムが対象とする課題はクラウドセンシングで重要である,依頼者側のサーバやアプリなどの専用システムの開発にかかるイニシャルコストやランニングコスト,依頼者の知識不足により発生する,危険なセンサデータの収集,協力者側のセンサデータ提供に対するディスインセンティブ要素である.

\section{センシング端末の課題}

クラウドセンシングに必要なセンサを搭載したセンシング端末にはいくつかの課題がある.
クラウドセンシングに専用のデータロガーを使用した場合の課題として,データロガーの確保や配布,回収などにコストがかかる点が挙げられる.
協力者に専用のデータロガーを持たせてセンシングする場合,協力者の人数分データロガーを確保する必要がある.
そして,協力者に専用のデータロガーを配布し,センシングが終了したらデータロガーを回収する必要がある.
また,協力者がデータロガーを紛失する可能性もある.
データロガーにセンサデータをアップロードする機能が搭載されている場合,回収する必要はないが,もう一度クラウドセンシングをする場合再度データロガーを確保する必要がある.

クラウドセンシングにモバイルアプリケーションを使用した場合の課題として,協力者の物理的コストと心理的コストが挙げられる.
協力者の物理的コストとして,協力者の手間と端末への負担が挙げられる.
協力者は複数のクラウドセンシングに協力すると協力した分だけ専用のアプリケーションをインストールしなくてはならない.
例えば,3人の依頼者がいた場合,専用のクラウドセンシングアプリケーションは3つある.
そのため,3つのクラウドセンシングにすべて協力するとなった場合,3つもアプリケーションをインストールしなければならない.
もっと協力するとなると,その分インストールしなければならないアプリケーションの数も増える.
それにより,協力者の負担も増加するといえる.
協力者の端末はセンシングのためにセンサを使用する.
一般的なモバイル端末において,センサの使用は端末のバッテリーを消費を早める.
協力者の端末がセンサデータをアップロードするまでにセンサログを保持する必要がある.
そのために端末の容量を圧迫してしまう.
協力者は依頼者にセンサデータをアップロードする必要がある.
センサデータをアップロードするのにモバイル通信を使用した場合,端末のモバイル通信量を圧迫してしまう.
協力者のアプリケーション内での操作やアプリケーションを使用する際のデータ通信量などの負担が多いと,協力者はアプリケーションを放置または削除してしまう.

協力者への心理的コストとして,第三者へのセンサデータ提供に対する不安や個人情報悪用の心配などのプライバシ意識が挙げられる.
普段から携帯しているモバイル端末をセンシング端末とした場合,協力者は普段の行動をセンシングされ,プライバシを侵害される可能性がある.
そのため,協力者は認識していない時空間でのセンシングやセンサの使用に心理的障壁を持つ.
また,協力者が認識しているセンサや時空間だとしても,大多数の協力者は専門家ではない為,センシングの危険性を考慮しきれない.

本研究が対象とする課題は,協力者の負担と,アプリ端末のデータ通信量及び心理的コストである.

% 協力者が提供したくないデータは送信しない,送信済みの場合は削除申請ができなければならない.

\section{研究目的}
\label{sec:thesis}

本研究では協力者のディスインセンティブ要素の軽減を目的とし,ユーザのセンシングの協力かつ継続を促進する.
そのために協力者の発生しうる物理的及び心理的コストの軽減を行う.
物理的コストの軽減として,協力者の操作や通知を最小限にし,センサデータアップロードはWi-Fi下で行う.
協力者の操作の低減として,協力者は1つのアプリケーションで複数のクラウドセンシングに参加できるようにする.
また,インストール等環境設定を除き,協力者の主な操作はセンシング依頼承諾画面に移動する時とセンシング依頼に承諾,拒否する時のみにする.
協力者の普段の利用を妨げないように時空間判定やセンシングはバックグラウンドで行う.
協力者が一度センシング依頼に承諾していた場合,時空間に進入した時自動でセンシングする.
協力者に対する通知を最小限に抑えるため,協力者が時空間に進入する可能性が高い場合のみ通知を送る.
心理的コストの軽減として,協力者がクラウドセンシングの内容に納得し,協力すると判断した場合のみセンシングを行う.
また,協力者のプライバシーの侵害を防ぐために,本アプリでアップロードされるセンサデータ等はすべて匿名化及び抽象化され,協力者は自身のセンサデータの削除及び削除申請ができる.
% 協力者の物理的コスト面の課題(依頼者側のサーバやアプリなどの専用システムの開発にかかるイニシャルコストやランニングコスト):個別アプリケーションのインストールをなくして,一つにする.
% また,スマートフォンの操作,通知を最小限にする.時空間に進入しないなら通知を出さない.一度承諾,拒否したらもう通知を出さない.


\section{論文構成}
\label{sec:presentation}

本稿の構成は以下の通りである.
2章では,クラウドセンシング及びクラウドセンシングプラットフォーム関連の既存研究を紹介し,その本研究との関連性を述べる.
3章では,時空間フェンシングを定義し,それに基づいたクラウドセンシングプラットフォームの全体像について述べる.
4章では,特定の時空間への進入時に自動センシングするアプリケーションの実装について述べる.
5章では,時空間フェンシングと実際のユースケースを想定した動作検証を行う.
6章では,まとめと今後の課題について述べる.

% 画像の置き方
% \begin{figure}[H]
%   \centering
%   \includegraphics[width=150mm]{cloudsensing.png}
%   \caption{クラウドセンシングによるデータ収集の意義}
%   \label{cloudsensing}
% \end{figure}
% 画像の呼び出し
% \ref{cloudsensing}

% Local Variables: 
% mode: japanese-LaTeX
% TeX-master: "root"
% End: 
