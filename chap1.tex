\chapter{はじめに}
\thispagestyle{myheadings}


\section{研究背景}
\label{sec:schedule}

近年,高機能センサを備えたスマートフォンが増加している.
近年高性能センサを搭載したスマートフォンが増加している.
クラウドセンシングは幅広いデータ収集かつセンシングコストを削減できるため、様々な研究で採用されている。
しかし、クラウドセンシングにはいくつかの課題があり、それを解決するために、我々は時空間フェンシングに基づくクラウドセンシングプラットフォーム「ラヴラス」を提案した。
% プラットフォームにして、クラウドセンシングの容易利用と多様なデータ収集ができるようにして、研究や調査におけるコスト(時間・費用・手間)を大幅に軽減する。
% 時空間フェンシングを提案して依頼者はセンシングする範囲を定義しやすく、協力者はセンシングされている時を明確に。
本研究はラヴラスのモバイルアプリケーションに関する研究である。
% クラウドセンシングプラットフォームを作ろうとしていて、自分はモバイルアプリ開発をしている。
% データ収集をしたい人(依頼者)はwebアプリから、センシングに協力する人(協力者)はスマホアプリから
% 依頼者が本プラットフォームを利用してセンシング依頼(センシングプロジェクト)を作成する。

\section{クラウドセンシングの課題}
依頼者側の課題として、サーバやアプリなどの専用システムの開発にかかるイニシャルコストやランニングコストが挙げられる。
また、依頼者の知識不足により、本来扱ってはいけない協力者のプライバシを侵害するセンサデータを集めてしまったり、協力者にセンシングがプライバシを侵害する危険性を説明しきれない可能性がある。
クラウドセンシングで協力者から集めたデータのクオリティが依頼者の要求するレベルに達しない場合がある。
協力者側の課題として、センサデータの提供にはディスインセンティブ要素が多い点が挙げられる.
% クラウドセンシングにはセンサデータを提供してくれる協力者が必須。
% 例えば協力者はいつセンシングされているかわからない。どのようにセンシングされているかわからない。
% プライバシを侵害されるような危険なセンシングをされている可能性がある。
本研究が対象とする課題はクラウドセンシングで重要である、依頼者側のサーバやアプリなどの専用システムの開発にかかるイニシャルコストやランニングコスト、依頼者の知識不足により発生する、危険なセンサデータの収集、協力者側のセンサデータ提供に対するディスインセンティブ要素である。

\section{センシング端末の課題}

クラウドセンシングに必要なセンサを搭載したセンシング端末にはいくつかの課題がある。

クラウドセンシングに専用のデータロガーを使用した場合の課題として、物理的コストが挙げられる。

% データロガーの確保や配布、回収
クラウドセンシングにモバイルアプリケーションを使用した場合の課題として、協力者の物理的コストと心理的コストが挙げられる。
協力者は複数のクラウドセンシングに協力すると協力した分だけ専用のアプリケーションをインストールしなくてはならない。
協力者のアプリケーション内での操作やアプリケーションを使用する際のデータ通信量などの負担が多いと、協力者はアプリケーションを放置または削除してしまう。
第三者へのセンシングデータ提供に対する不安や個人情報悪用の心配などのプライバシ意識により協力者獲得は容易ではない。
% 協力者が提供したくないデータは送信しない、送信済みの場合は削除申請ができなければならない。

\section{研究目的}
\label{sec:thesis}

本研究では協力者のディスインセンティブ要素の軽減を目的とし、ユーザのセンシングの協力かつ継続を促進する。
そのために協力者の発生しうる物理的及び心理的コストの軽減を行う。
% 協力者の物理的コスト面の課題(依頼者側のサーバやアプリなどの専用システムの開発にかかるイニシャルコストやランニングコスト):個別アプリケーションのインストールをなくして、一つにする。
% また、スマートフォンの操作、通知を最小限にする。時空間に進入しないなら通知を出さない。一度承諾、拒否したらもう通知を出さない。
センシングデータアップロードはWi-Fi下で行う。
心理的コスト面の課題:センシングデータ提供に対する不安は、依頼者の情報を提示して、センシング依頼に承諾してもらう。不安があればセンシング依頼に承諾した後でも拒否できる。未送信のセンシングデータは削除ができ、送信済みなら削除申請が出せる。
また、協力者のプライバシーの侵害を防ぐために、本アプリでアップロードされるセンサデータ等はすべて匿名化及び抽象化する。


\section{論文構成}
\label{sec:presentation}

% 本稿の構成は以下の通りである.
% 2章では,クラウドセンシング及びクラウドセンシングプラットフォーム関連の既存研究を紹介し,その本研究との関連性を述べる.
% 3章では,時空間フェンシングを定義し,それに基づいたクラウドセンシングプラットフォームの要求仕様・実装について述べ,動作検証を行う.
% 最後に4章では,まとめと今後の課題について述べる.

% 画像の置き方
% \begin{figure}[H]
%   \centering
%   \includegraphics[width=150mm]{cloudsensing.png}
%   \caption{クラウドセンシングによるデータ収集の意義}
%   \label{cloudsensing}
% \end{figure}
% 画像の呼び出し
% \ref{cloudsensing}

% Local Variables: 
% mode: japanese-LaTeX
% TeX-master: "root"
% End: 
