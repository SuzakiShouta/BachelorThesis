\chapter{はじめに}
\thispagestyle{myheadings}


\section{研究背景}
\label{sec:schedule}

近年,高機能センサを備えたスマートフォンが増加している.
内蔵されているセンサの例として,加速度や角速度,磁気,気圧,Wi-Fi,BLE,音などが挙げられる.
これらのセンサは,正確で高精度な測定データの取得が可能なため,スマートフォンの3次元での位置や動作,周囲の環境のモニタリングが可能である.
例えば,ユーザの現在位置や動き,ゲームでは,スマートフォンを振る・傾けるなどの複雑な操作や動きの測定が内蔵センサによって可能となる.
追加モジュールによっては,花粉やPM2.5,風速や湿度などの計測も可能となる.

\begin{figure}[H]
  \centering
  \includegraphics[width=150mm]{cloudsensing.png}
  \caption{クラウドセンシングによるデータ収集の意義}
  \label{cloudsensing}
\end{figure}


このスマートフォンのセンシング能力を活かす試みとして,クラウドセンシングがある.
クラウドセンシングの説明図を図\ref{cloudsensing}に示す.
クラウドセンシングとは「群衆の持つスマートフォンなどの携帯端末に内蔵されたセンサを用いて低コストで大規模にセンサデータ取得し,そこから実世界の様相を把握するための方法論\cite{weko}」である.
クラウドソーシングとセンシングを掛け合わせたものである.
参加型センシングとも言われる.
スマートフォンの登場以前は,センサデータを多くの人から取得するのは現実的ではなかった.
クラウドセンシング専用のデータ収集端末を開発しても,収集したデータの共有は困難であった.
スマートフォンは文字入力,センシング,通信機能など様々な機能を備えているため,クラウドセンシングを行うための端末として非常に適している.
スマートフォンの登場によって,クラウドセンシングという考え方が進んだと言っても過言ではない.
クラウドセンシングの技術的背景としては,スマートフォンの内蔵センサの小型化・高性能化が挙げられる.
センサの精度は初期のものと比べ,格段に良くなっている.
また,どんどん小型化・スマート化し,電力消費量も少なくなっている.
もう1つの技術的背景として,通信速度の高速化が挙げられる.
現在は4G(第4世代移動通信システム)が主流となっているが,2020年には5G(第5世代移動通信システム)のサービスが開始された.
5Gの通信速度は4Gの20倍,遅延は4Gの10分の1,4Gの10倍のデバイスに接続が可能となる\cite{5G4G}.
クラウドセンシングの社会的背景としては,スマートフォンの急速な普及が挙げられる.
2007年にApple社から「iPhone」,2008年にはAndroid端末が発売され,全世界に「スマートフォン」が爆発的に普及されていった.
日本でも主流は「フィーチャーフォン」から「スマートフォン」に変化した.
2010年はスマートフォンの世帯保有率は9.7%\cite{soumu}で,まだ全体的な普及はしていなかった.
この時点ではクラウドセンシングに協力するしないにかかわらず,まずクラウドセンシングに協力するためのスマートフォンを所持しているかしていないかで対象者が振り分けられる.
つまり,対象者の母数が圧倒的に少なかった.
クラウドセンシングは収集できるデータ数が多ければ多いほど価値が高くなるため,スマートフォンを所持していない2010年頃はクラウドセンシングを利用するには適していなかった.
2019年にはスマートフォンの保有率は世帯では83.4%と2010年と比べ約9倍,個人では67.6%\cite{soumu}となり,半数以上の人がスマートフォンを所持するようになっている.
所持している人全員がクラウドセンシングに協力してくれるとは限らないが,明らかに母数は増えているため,現在はクラウドセンシングを利用するのに適している.

クラウドセンシングは幅広いデータ収集かつセンシングコストを削減できるため,様々な研究で採用されている.
例えば,街頭の明るさ調査,観光スポット調査,騒音調査,温度調査などの研究が進められている\cite{liu}.
クラウドセンシングは研究だけではなく,自治団体や地域施設の管理者など,様々な人にとって有効になりうる可能性がある.
例えば先述の騒音調査や公害物質調査などは,自治団体にとって地域の住みやすさ改善のための重要なデータとなり,そのデータを基によりよい地域や町づくりが可能となる.
また,公園管理者にとってはいつどのような場所に人が集まるのか,危険な箇所で子供が遊んでいないか,といった施設の安全確保に有効なデータを収集できる.

% -------------------------------------------------------------------------------
\section{クラウドセンシングの課題}
\label{sec:abstract}
幅広いデータ収集かつセンシングコストを削減できるクラウドセンシングにはいくつかの課題がある.
大きく分けて依頼者側・協力者側の課題がある.
本研究では,クラウドセンシングを利用しデータ収集を行う人を依頼者,クラウドセンシングに協力しスマートフォンでデータ提供を行う人を協力者とする.

依頼者側の課題として,サーバやアプリなど専用システムの開発にかかるイニシャルコストやランニングコストが挙げられる.
依頼者がクラウドセンシングを利用するためには,協力者のセンサデータを収集するための専用アプリや収集したセンサデータを管理するための専用サーバが必要となる.
自治体や地域施設の管理者などシステムの開発知識がない人にとっては自ら開発するのは困難であり,業者に委託するにも費用や手間などが大きくかかる.
また,知識のある研究者にとっても,システムの開発という研究の本質からずれた作業をしなければならないため,研究の速度は下がってしまう.
そのため,依頼者のクラウドセンシングに対するハードルは高くなっている.

協力者側の課題として,物理的及び心理的コストといった課題が挙げられる.
物理的コストとしては個別のクラウドセンシングアプリケーションのインストールや利用などにかかる手間や協力者の操作や通信といった負担が挙げられる.
心理的コストとしては協力者のプライバシ障壁によるデータ提供への心配・不安,それらによるストレスなどが挙げられる.

まず協力者がクラウドセンシングに参加するためには,個別のクラウドセンシングアプリケーションをインストールしてもらい,継続的に利用してもらう必要がある.
多くの人にとって,個別のアプリケーションのインストールはコストが高い.
例えば,3人の依頼者がいた場合,専用のクラウドセンシングアプリケーションは3つある.
そのため,3つのクラウドセンシングにすべて協力するとなった場合,3つもアプリケーションをインストールしなければならない.
もっと協力するとなると,その分インストールしなければならないアプリケーションの数も増える.
それにより,協力者の負担も増加するといえる.
また,継続的に個別のアプリケーションを利用してもらうのも困難である.
多くの人々は「一定期間利用しないアプリケーションは削除したい」といった欲求があるため,一定期間クラウドセンシングを行わないとアプリケーションを削除されてしまい,いざデータを収集したいとなったときには出来ない可能性がある.

また,協力者のアプリケーション内での操作やアプリケーションを使用する際のデータ通信量などの負担が多いと,協力者はアプリケーションを放置または削除してしまう.
たとえ多くの協力者の獲得できたとしても,アプリケーションの操作や設定が複雑であったり,センシングデータの通信で協力者のデータ通信量を多く使ってしまったりする場合,協力者にとってクラウドセンシングが億劫に感じてしまうため,多くの収集データ量は見込めない.
そのため,センシングやセンシングデータアップロードのために必要なスマートフォン操作や通信コストを最小限に抑える必要がある.

そして,第三者へのセンシングデータ提供に対する不安や個人情報悪用の心配などのプライバシ意識により協力者の獲得は容易ではない.
もし,依頼者と協力者が信頼しあっており,データ提供に対する不安がなければ,センシングに対する協力は容易である.
しかし,依頼者と協力者が赤の他人の関係であったら,協力者は自分自身の個人情報が悪用されるのではないかと疑ってしまい,データ提供の可能性は低い.
また,協力者は依頼者を信頼していても,データ提供してもいい時とデータ提供したくない時がある.
そのため,ある程度センシングする範囲や時間帯を絞って,協力者の承諾を得る必要がある.

また,クラウドセンシングではスマートフォンのセンサ情報や位置情報などの個人情報を多く取り扱うため,セキュリティやプライバシ保護の対策も万全である必要がある.
依頼者と協力者の信頼関係が成り立っていても,データ管理のセキュリティが甘いと,第三者に悪用されてしまう場合がある.
そのため,クラウドセンシングを利用する際には,セキュリティやプライバシ保護への対策ができている必要がある.



\section{研究目的}
\label{sec:thesis}

本研究ではクラウドセンシングの簡易的な利用と多様なデータ収集を行い,研究や調査におけるイニシャル及びランニングコストの大幅な軽減を目的とする.
依頼者側で課題となっていた専用システムを個別で開発する必要をなくし,センサの種類を選択するだけで簡易的にクラウドセンシングが利用可能となるシステムを実現する.
また,加速度センサのみなどの単一のセンサデータだけでなく,“加速度センサと角速度センサ”や“加速度センサと気圧センサとWi-Fi”など多様なセンサデータの収集により,どの研究や調査にも柔軟に対応する.

そして,協力者の発生し得る物理的及び心理的コストの軽減し,ユーザのセンシングの協力かつ継続を促進する.
協力者側の物理的コスト面での課題となっていた個別アプリケーションのインストールをなくし,共通のアプリケーション1つでどのクラウドセンシングにも協力可能にする.
それにより,協力者のインストールに対する負担を軽減し,クラウドセンシングの依頼が頻繁に行われていればアプリケーションが削除される心配もなくなる.
また,スマートフォンの操作を最小限にし,センシングデータアップロードはWi-Fiなど携帯回線以外に接続されている場合のみに行う.
心理的コスト面での課題となっていたセンシングデータ提供に対する不安は,依頼者の情報や目的や使用するセンサの種類などのセンシング内容の提示により払拭する.
依頼者やセンシング内容に納得しセンシングの依頼を承諾した場合のみのセンシング開始により,勝手にセンシングされたり,センシングされたくないプライベートな時間にセンシングされるなどがなくなる.
これにより,協力者のセンシングに対するハードルを下げられる.

我々はこの目的を実現するためにまず時空間フェンシングの概念を提案する.
時空間フェンシングとは,地理的な場所を制限するジオフェンシングに時間要素を加えて拡張し,時間と場所で境界を区切る独自のフェンシング手法である.
これにより,依頼者はシチュエーションを指定したデータ収集が可能となる.
協力者はセンシングを承諾するか否かの判断がしやすくなる.

そして,この時空間フェンシングに基づいたクラウドセンシングプラットフォームを構築する.
依頼者はWebアプリケーションでセンシング依頼の定義を行い,協力者はスマートフォンアプリケーションでセンシングを行う.
これにより,依頼者は時間帯やエリア,目的,概要,センサの種類など必要な情報の入力のみでクラウドセンシングが可能となる.
また,データ収集にかかるコスト削減かつ柔軟なセンシングが可能となる.
協力者のセンシング協力には共通のスマートフォンアプリケーション1つでいいため,インストールのコストを軽減できる.
さらに,依頼者が新規のクラウドセンシングを行う際に新規の協力者を募らなくても,既に多くの協力者が本プラットフォームのスマートフォンアプリケーションを利用している状況になるため,協力者獲得が容易となる.

    % ビーコン番号 & 参加型10\% & 20\% & 30\% & 無意識型  \\ \midrule
%    1(道幅25m) & 24 & 42 & 71 & 238 \\ 
  %  2(道幅20m) & 22 & 37 & 76 & 172  \\
    %3(道幅15m) & 16 & 29 & 52 & 71  \\ 
    %4(道幅5m) & 28 & 45 & 78 & 39  \\ \bottomrule

\section{論文構成}
\label{sec:presentation}

本稿の構成は以下の通りである.
2章では,クラウドセンシング及びクラウドセンシングプラットフォーム関連の既存研究を紹介し,その本研究との関連性を述べる.
3章では,時空間フェンシングを定義し,それに基づいたクラウドセンシングプラットフォームの要求仕様・実装について述べ,動作検証を行う.
最後に4章では,まとめと今後の課題について述べる.

% Local Variables: 
% mode: japanese-LaTeX
% TeX-master: "root"
% End: 
