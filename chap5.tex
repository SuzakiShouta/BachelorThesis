\chapter{動作検証}
\thispagestyle{myheadings}
本研究の動作検証は特定の時空間に進入時のみセンシングできているか、プラットフォームとして複数のユースケースを想定して適切にセンシングできているかの2つを行う。

\section{時空間フェンシングの動作検証}

\section{ユースケースを想定した動作検証}
天候によって所要時間が変化する地図アプリを作成したい人がいたとする。
研究室の管理者が、研究室内でどれだけコミュニケーションが取れているか測定しようとしたとする。


% Local Variables: 
% mode: japanese-LaTeX
% TeX-master: "root"
% End: 
