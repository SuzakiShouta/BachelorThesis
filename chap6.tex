\chapter{おわりに}
\thispagestyle{myheadings}

\section{まとめ}
本研究では協力者のディスインセンティブ要素を軽減し,ユーザのセンシングの協力かつ継続を促した.
動作検証では時空間フェンシングが適切に行えているか確認した.その結果時空間フェンシングが適切に行えた.
実際のユースケースを想定して適切にセンシングできているか検証した.その結果さまざまなセンサデータを集められた.

\section{今後の課題}
今後の課題として,今回実装に至らなかった時空間フェンシングに基づくクラウドセンシングプラットフォームにおけるモバイルアプリケーションに必要な機能の実装が挙げられる.
時空間フェンシングにGPSを使用しているのでGPSの精度が落ちる屋内で動作が不安定になる点.


% Local Variables: 
% mode: japanese-LaTeX
% TeX-master: "root"
% End: 
