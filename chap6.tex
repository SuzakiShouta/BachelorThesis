\chapter{おわりに}
\thispagestyle{myheadings}

\section{まとめ}
本論ではクラウドセンシングプラットフォームにおけるモバイルアプリに求められる機能の要件定義をした.

その中から,時空間フェンシング,センシング依頼,自動でセンシングを実装した.
時空間フェンシングの実装ではジオフェンスが任意の多角形である場合に対応するため,ポリゴンの内外判定を使用した.
確実に時空間に進入した場合のみセンシングする,時空間に進入する可能性が高い協力者にセンシング依頼通知を発行するなど,様々なシチュエーションに対応するため,時空間の拡大と縮小が可能なマージンを実装した.
また,ジオフェンスが複雑な矩形である場合に対応する為,協力者の位置情報にマージンを設けた.
センシング依頼の実装ではまずセンシング依頼に参加する可能性が高い協力者を定義した.
次にセンシング依頼に参加する可能性が高い協力者にのみ通知を発行する.
また,協力者がセンシングに参加するか否かを判断するセンシング依頼画面を実装した.
センシング依頼に承諾し,時空間に進入した場合自動でセンシングする機能を実装した.
確実に協力者の時空間への進入と退出を判定するため,進入時は時空間を狭くするマージン,退出時は時空間を広くするマージンを取った.
クラウドセンシングプラットフォームとして多くのセンサと自由な周波数に対応し,プライバシを侵害するセンサデータは抽象化した.

動作検証では時空間フェンシングが適切に行えているか確認した.その結果時空間フェンシングが適切に行えた.
また,実際のユースケースを想定して適切にセンシングできているか検証した.その結果さまざまなセンサデータを集められた.

\section{今後の課題}
今後の課題として,今回実装に至らなかった時空間フェンシングに基づくクラウドセンシングプラットフォームにおけるモバイルアプリケーションに必要な機能の実装が挙げられる.
今回実装できなかったセンシングプロジェクトダウンロードやセンサデータアップロードなどサーバとの連携を実装する必要がある.

時空間フェンシングの課題として,時空間フェンシングにGPSを使用しているのでGPSの精度が落ちる屋内で動作が不安定になる点が挙げられる.
現時点でジオフェンシングに緯度経度のみを使用しているため,GPSの精度が落ちる屋内や高層ビルが多くある場所ではジオフェンスが適切に行えない.
また,緯度経度のみでは1階と2階の判定もできない.
そのため,緯度経度でのジオフェンシング以外にBLEやWi-Fiなどの実装が必要である.

クラウドセンシングのプラットフォームとして,協力者の基礎情報登録が挙げられる.
依頼者によっては性別や身長とセンサデータを結びつけて調査を行うケースも考えられる.
そうした場合,協力者に性別や身長など自身の基礎情報をスマホアプリインストール時に登録してもらう必要がある.
基礎情報の登録機能を実装する場合,自身の性別や身長を提供したくない協力者への登録強制はせず,任意での登録とする.

その他の課題として,マイクやカメラのセンシングデータの値の抽象化や行動認識モジュールによる収集,スマートウォッチへの対応などが挙げられる.
マイクやカメラによって収集した音声・映像データは,プライバシに関わる情報が非常に多く含まれてしまっている.
依頼者にとっても,音声・映像データの取り扱いは困難である.
そのため,プラットフォーム側で信号処理モジュールを通して発言数,騒音レベル,明るさレベル,といった値に抽象化した上で依頼者側にダウンロードしてもらう必要がある.
加速度等,他のセンサ信号についても,歩数や消費カロリーといった多くの需要が想定される値に行動認識モジュールによって可能にする.
スマートフォンだけではなくスマートウォッチにも対応すると,より柔軟で多様なデータ収集が可能となる.
例えば,スマートウォッチのセンサからバイタルデータを収集して,運動負荷の調査などが可能となる.

実際に協力者のディスインセンティブ要素を軽減し,ユーザのセンシングの協力かつ継続ができたかを評価実験する必要がある.
評価実験は今後このラヴラスを運用するにあたって,とても重要なものである.
本アプリの目標は協力者の増加であるため,多くのユーザの使用してもらう必要がある.
そして,協力者から実際にクラウドセンシングの参加に対する負担が少なかったかどうかのフィードバックをもらい,アプリを修正する必要がある.


% Local Variables: 
% mode: japanese-LaTeX
% TeX-master: "root"
% End: 
