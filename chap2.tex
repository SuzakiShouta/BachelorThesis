\thispagestyle{myheadings}
\chapter{関連研究}
\label{sec:format}

% 本章では本研究の関連研究について述べる.
% 2.1では,スマートフォンを用いたセンシングとジオフェンシングの関連研究について述べる.
% 2.2では,1章で説明したクラウドセンシングの関連研究について述べる.
% 2.3では,クラウドセンシングの基盤となるプラットフォームの運用例や関連研究,それらの協力者に対するモチベーションの維持・向上方法について述べる.

\section{クラウドセンシングに関する研究}
\label{sec:format_thesis}
幅広いデータ収集かつセンシングコストを削減できるクラウドセンシングを利用している研究はいくつかある。
これらの研究ではクラウドセンシングシステムの開発などには大きなコストがかかると考えられる。


\section{クラウドセンシングプラットフォームに関する研究}
\label{sec:style}
実際に運用を行っているクラウドセンシングプラットフォームとして、OhmageやAWAREなどがある。
クラウドセンシングは協力者の確保が非常に重要であるため、様々な方法でモチベーションを向上・維持させる必要がある。
本研究ではディスインセンティブ要素を軽減する。


\section{センシング端末に関する研究}
\label{sec:format_abst}
クラウドセンシングのセンシング端末として様々な端末が使用されている。
例えば、スマートフォンが使用されている。
スマートフォンのクラウドセンシングは協力者がそのクラウドセンシングに対応したアプリケーションをそれぞれインストールする必要がある。
スマートフォンを使用しない例として、市販の環境センサや、専用に開発されたものがある。
% オムロン(IoTセンシングによる賃貸物件快適度推定のためのデータ収集)
% アメダス、日本海溝海底地震津波観測網
スマートフォンを使用せず、市販の環境センサや、専用に開発したものは、長時間のセンシングや大規模なセンシングが可能であるが、イニシャルコストとランニングコストがかかる。

% Local Variables: 
% mode: japanese-LaTeX
% TeX-master: "root"
% End: 
