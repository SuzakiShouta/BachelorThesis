\chapter{特定の時空間への進入時に自動センシングするアプリケーション}
\thispagestyle{myheadings}

\section{ラヴラスのモバイルアプリケーションの要求仕様}
ラヴラスのモバイルアプリケーションはセンシングプロジェクトダウンロード、時空間フェンシング、センシング依頼の承諾、自動的にセンシング、Wi-Fi環境下で自動的にアップロードの順で行う。
協力者が本アプリを起動、もしくは起動してから一定時間毎にサーバからセンシングプロジェクトをダウンロードする。
% あいまいな位置情報
協力者が時空間に進入した場合、通知が発行され、センシング依頼画面が立ち上がる。センシング依頼に承諾した場合、時空間に進入している間、バックグラウンドで自動でセンシングされる。
センシングが終わった後、Wi-Fiに接続している時に自動でアップロードされる。
協力者はすでにセンシングに承諾したセンシングプロジェクトにもセンシング拒否ができる。
% また、送信したセンサデータに削除申請ができる。
クラウドセンシングプラットフォームとして、多くのセンサに対応する必要がある。

\section{時空間への進入時に自動センシングするアプリケーションの実装}
依頼者の制作したセンシングプロジェクトに対応したセンシングをするためにAndroidアプリを作成した。
本アプリは4.1章で述べた内、時空間フェンシング、センシング依頼の承諾、自動的にセンシングのみ実装した。
\subsection{時空間フェンシングの実装}
時空間に進入しているかの判定のため、一定間毎に位置情報を現在時刻を取得する。
複雑な矩形に対応するためにポリゴンの内外判定アルゴリズムを使用する。
ジオフェンシングの境界付近かつ、位置情報が不安定になると進入、退出の判定を繰り返してしまう。これを防ぐためにマージンを設けた。
% 複雑な矩形のマージンに対応できるように端末にマージンを設けた。

\subsection{センシング依頼通知の実装}
協力者が時空間に進入するとセンシング依頼の通知が発行される。
協力者が安心してセンシングに協力できるように

\subsection{自動センシングの実装}
協力者が時空間に進入し、センシング依頼に承諾している場合、バックグラウンドで自動でセンシングされる。
クラウドセンシングプラットフォームとして多くのセンサと自由な周波数に対応した。


% Local Variables: 
% mode: japanese-LaTeX
% TeX-master: "root"
% End: 
