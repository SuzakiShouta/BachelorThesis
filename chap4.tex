\chapter{特定の時空間への進入時に自動センシングするアプリケーション}
\thispagestyle{myheadings}
本章ではまず時空間フェンシングの概念を定義し、次に時空間フェンシングに基づくクラウドセンシングプラットフォームの全体図について述べる。
本クラウドセンシングプラットフォーム「Lavlus」(以下,ラヴラス) の命名は、”a view of Laplace’sdemon”「ラプラスの魔の視界」から来ている。

\section{ラヴラスのモバイルアプリケーションの要求仕様}
時空間フェンシングは「ジオフェンシングに時間要素を追加し拡張したフェンシング手法」として定義する。
時空間フェンシングのメリットとして、時間とエリアで境界を区切ると依頼者は様々なシチュエーションを指定したクラウドセンシングが可能となる。
% (依頼者はクラウドセンシングの範囲を定義しやすい,依頼者も協力者もセンシングする場所を認識しやすい.)←これもメリット
協力者のクラウドセンシングに対するプライバシ障壁は、時空間フェンシングによる時間と空間の制限で軽減できる
% プライバシ障壁が軽減できるわけではなく,判断しやすい.センシング依頼承諾と合わせていい感じになる.
時空間フェンシングのデメリットとして、時間と空間に依存しないクラウドセンシングに適さない点である。
% 例えば、空間に依存しない移動する電車内でのクラウドセンシングや、時間に依存しない雨が降った時のみのクラウドセンシングなど。

\section{時空間への進入時に自動センシングするアプリケーションの実装}

\subsection{時空間フェンシングの実装}

\subsection{センシング依頼通知の実装}

\subsection{自動センシングの実装}


% Local Variables: 
% mode: japanese-LaTeX
% TeX-master: "root"
% End: 
