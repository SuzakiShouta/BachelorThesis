\chapter{おわりに}
\thispagestyle{myheadings}

\section{まとめ}
本稿では初めに時空間フェンシングについて定義した.
時空間フェンシングとは,地理的な場所を制限するジオフェンシングに時間要素を加えて拡張し,時間と場所で境界を区切る独自のフェンシング手法である.
時空間フェンシングによって時間とエリアで境界を区切ると,依頼者は様々なシチュエーションを指定したクラウドセンシングが可能となる.
一方で,時間やエリアに依存していないデータ収集には適していない.

次に,時空間フェンシングに基づいたクラウドセンシングプラットフォームの要求仕様を定義した.

依頼者がセンシングデータや各プロジェクトを管理するサーバを実装した.

依頼者がセンシングプロジェクトの作成やセンシングデータのダウンロードなどを行うWebアプリケーションを実装した.

協力者が時空間フェンシングに基づいたセンシングに協力するためのスマートフォンアプリケーションを実装した.

動作検証では,本プラットフォームの一連の流れが正常に動作するか検証した.
\section{今後の課題}
まず今後の課題としては,今回完全に実装できなかった協力者のプライバシ保護の徹底が挙げられる.

プラットフォームの運用を行うための課題として利用規約の作成をする必要がある.
利用規約には,要求仕様の3.2.3「ラヴラスの利用及び利用規約の方針」に示したセンシングデータの保存期間以上の保存の禁止などの項目などに加え,利用に関しての前提条件の説明やプラットフォームが負う責任の詳細など制定し実際の運用に対応させるとともに,利用規約を前提として今後の実装を行うためである.

時空間フェンシングの課題として,時空間フェンシングの最大範囲の最適値の模索がある.
要求仕様の3.2.1「クラウドセンシングプラットフォームとしての設計基盤」に時空間フェンシングの最大範囲を設定すると述べた.
これは,設定する時空間を無制限に拡大してしまうと時空間フェンシングの意義がなくなってしまうからである.
しかし,ラヴラスでは,現在,評価実験を行えていないため最大範囲をどこまでとするのかの指標がないため適切な時空間フェンシングの最大範囲の設定ができていない.
この課題は,時空間フェンシングによって制限される時空間がどの範囲までフェンシングとして有効なのかを評価実験によって調査した上で最大範囲を設定する必要がある.

プラットフォーム全体の実装面における課題として,GPS以外のでエリア判定を可能にするシステムを導入,参加済みのセンシングプロジェクトの辞退や削除要請の実装,センシングデータの軽量化,センシング設定のプリセットの作成がある.
GPS以外でのエリア判定を導入する理由は,現在採用しているGPSによるエリア判定では,狭い範囲でのエリア範囲やGPSの不安定な屋内の判定,更には屋内における階数による判定や部屋単位のエリア判定ができないためである.
これらの判定は,Wi-FiやBLEといった電波を用いたエリア判定や建物から発生する磁気を用いたエリア判定の導入により実現する.
GPS以外でのエリア判定を実装するには,まずサーバにGPS以外のエリア判定を定義できるモデルを作成した上でWebアプリのGPS以外のエリア判定の設定項目の追加,そしてスマホアプリにGPS以外の判定アルゴリズムを作成する必要がある.
参加済みのセンシングプロジェクトの辞退や削除要請の実装は,要求仕様として3.2.2「プライバシの保護」に定義を行ったものの現在実装に至っていない.
この課題は,ラヴラスの運用にあたって必須の実装であるため優先的に解決する必要がある.
センシングデータの軽量化は,現在採用しているセンシングデータのフォーマットでのセンシングデータが肥大化する傾向にあるため軽量化によって要求仕様の3.2.4「利用者への配慮」に挙げた通信量に関する物理的負担を軽減するとともに,リアルタイムでのセンシングを実装を視野に入れる.
現在採用されているセンシングデータのフォーマットは,すべての情報をテキストにて表現を行っているためセンシングデータの特性を考慮するとセンシングデータは数字による表現が多いためテキストでなく数値を表現するバイナリを使用した上で,センシングデータの圧縮によって大幅なセンシングデータの削減を実現できると考えられる.
センシング設定のプリセットは,多くの需要が想定される加速度のセンシングを利用した歩数や消費カロリーなどのセンシングをプリセットとして定義し,3.2.4「利用者への配慮」に挙げたユーザインターフェイスの向上を目的とし,クラウドセンシングの定義をより簡略化が可能となる.

サーバの課題として,依頼者情報の信頼性の向上が挙げられる.
これは,要求仕様の3.2.3「ラヴラスの利用及び利用規約の方針」に定義した依頼者情報の明示が依頼者によって保証されるためプラットフォームとして依頼者情報を保証していない.
そのため,依頼者の設定したメールアドレスアドレスが組織や機関のものかどうかなどの確認を行うなどで依頼者情報の信頼性を向上させる目的させる必要がある.

Webアプリの課題として,Webアプリ上でのセンシングデータの可視化が挙げられる.
ラヴラスは,将来的に自治団体や地域施設の管理者といった依頼者の利用を想定している.
現在のセンシングデータの提供方法は,センシングデータのダウンロードという形で行っているが,こういった依頼者によってセンシングデータのダウンロード・解析は困難であると想定される.
そのため,Webアプリ上でのセンシングデータの可視化により,センシングデータの解析が困難な依頼者にとってもクラウドセンシングを実施できるプラットフォームを実現する必要がある.

スマホアプリの課題として,協力者の基礎情報登録が挙げられる.
依頼者によっては性別や身長とセンサデータを結びつけて調査を行うケースも考えられる.
そうした場合,協力者に性別や身長など自身の基礎情報をスマホアプリインストール時に登録してもらう必要がある.
基礎情報の登録機能を実装する場合,自身の性別や身長を提供したくない協力者への登録強制はせず,任意での登録とする.



その他の課題として,マイクやカメラのセンシングデータの値の抽象化や行動認識モジュールによる収集,スマートウォッチへの対応などが挙げられる.
マイクやカメラによって収集した音声・映像データは,プライバシに関わる情報が非常に多く含まれてしまっている.
依頼者にとっても,音声・映像データの取り扱いは困難である.
そのため,プラットフォーム側で信号処理モジュールを通して発言数,騒音レベル,明るさレベル,といった値に抽象化した上で依頼者側にダウンロードしてもらう必要がある.
加速度等,他のセンサ信号についても,歩数や消費カロリーといった多くの需要が想定される値に行動認識モジュールによって可能にする.
スマートフォンだけではなくスマートウォッチにも対応すると,より柔軟で多様なデータ収集が可能となる.
例えば,スマートウォッチのセンサからバイタルデータを収集して,運動負荷の調査などが可能となる.

これらの課題を解決した上で,依頼者・協力者ともにより柔軟に,より利用しやすくするため,評価実験の実施が必要がある.
評価実験は今後このラヴラスを運用するにあたって,とても重要なものである.
現在,基礎的な実装は動作検証にて確認できているものの,実際に依頼者と協力者に利用してもらったわけではない.
そのため,今までの要求仕様を基にした実装が利用ユーザにとって有益なものかは定かではない.
依頼者にとってクラウドセンシングがより簡易的に,より柔軟にできているか,協力者にとってクラウドセンシングに負担が少なく協力できているかを評価してもらい,再構築を行う必要がある.
評価実験を行うにあたって,プライバシの保護と個人情報の対策を万全にする必要がある.
万全ではない状態で行うと,万が一不具合が起こった場合にユーザの個人情報を流出する恐れがあり,ユーザから信用を失う可能性があるため,評価実験は慎重に行う必要がある.
評価実験を基に,依頼者側ではより多くのデータを集め,協力者側ではより安心してより簡易的にセンシングを可能にする必要がある.
% Local Variables: 
% mode: japanese-LaTeX
% TeX-master: "root"
% End: 
